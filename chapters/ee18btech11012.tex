\begin{enumerate}[label=\thesubsection.\arabic*.,ref=\thesubsection.\theenumi]
\numberwithin{equation}{enumi}

\item Consider an op amp having a single pole open loop response $G_{o} = 10^5$ and $f_{p} = 10$ Hz.Let op amp be ideal connected in non-inverting terminal with a nominal low frequency of closed loop gain of 100 and wired as a unity gain buffer.
\subitem Find the frequency at which $|GH| = 1$ and 
What is its corresponding phase margin
\solution  
For a single-pole amplifier, open loop transfer function is 

\begin{align}
    G\brak{s} = \frac{G_{o}}{1+\frac{s}{\omega_{p}}}
\end{align}
Given that $f_{p} = 10$ Hz and $G_{o} =10^5$
\begin{align}
G\brak{s}=\frac{G_{o}}{1+\frac{s}{2\pi f_{p}}}
\implies \frac{10^{5}}{1+\frac{s}{2\pi.10}}
\end{align}
So,the open-loop gain of the op amp is 
\begin{align}
    G\brak{s}=\frac{10^{5}}{1+\frac{s}{2\pi.10}}
\end{align}
For a unity-gain buffer,the feedback factor is
\begin{align}
    H = 1
\end{align}
Thus, 
\begin{align}
    G\brak{\j\omega}H = \frac{10^{5}.1}{1+\frac{\j\omega}{2\pi.10}}
\end{align}
To find the frequency at which $|G(j\omega)H|=1$ , we write
\begin{align}
    |\frac{10^{5}.1}{1+\frac{\j\omega}{2\pi.10}}| = 1
\end{align}
\begin{align}
    {1+\frac{\omega_{1}^2}{2\pi.10}} = 10^{10}
\end{align}
Thus  
\begin{align}
    \omega_{1} = 6.283 Mrad/sec
\implies f_{1} =\frac{\omega_{1}}{2\pi}=1 MHz
\end{align}
From definition of phase margin $\alpha = 180\degree + \phi$
where $\phi$ is the phase of $G(j\omega_{1})H$
\begin{align}
\phi = -\tan^{-1}\brak{\frac{\omega_{1}}{2\pi.10}}
\label{eq:ee18btech11012_phaseGH}
\end{align}
At $\omega_{1} = 2\pi.10^{6}rad/sec$
\begin{align}
    \phi = -\tan^{-1}\brak{{2\pi.10^{6}}{2\pi.10}} \\
\implies \phi = -90\degree (approx)
\end{align}
Therefore,the phase margin is
\begin{align}
    \alpha = 180\degrqq + \phi \implies \alpha = 180\degree - 90\degree\implies  \alpha = 90\degree
\end{align}
\textbf{Hence for frequency $f = 1 MHz$ Hz, $|GH| = 1$ and phase margin is 90\degree}
\item The following is the code for bode plot of the given system
\begin{lstlisting}
codes/ee18btech11012_1/ee18btech11012_1.py
\end{lstlisting}
\item Verification using Bode plot 
\begin{figure}[!h]
\centering
\includegraphics[width=\columnwidth]{./figs/ee18btech11012_1/ee18btech11012_1.eps}
\caption{}
\label{fig:ee18btech11012_1}
\end{figure}
\item Realise the above system using a feedback circuit.\\
\solution
\begin{figure}[ht!]
	\begin{center}
		\resizebox{\columnwidth/1}{!}{\tikzstyle{block} = [draw, rectangle, 
    minimum height=1.25em, minimum width=2.5em]
\tikzstyle{sum} = [draw, circle, node distance=1cm]
\tikzstyle{input} = [coordinate]
\tikzstyle{output} = [coordinate]
\tikzstyle{pinstyle} = [pin edge={to-,thin,black}]

% The block diagram code is probably more verbose than necessary
\begin{tikzpicture}[auto, node distance=2.5cm,>=latex']
    % We start by placing the blocks
    \node [input, name=input] {};
    \node [sum, right of=input] (sum) {};
    \node [block, right of=sum] (controller) {$\frac{10^{5}}{1+\frac{s}{2\pi \times 10}$};
    
    % We draw an edge between the controller and system block to 
    % calculate the coordinate u. We need it to place the measurement block. 
   
    \node [output, right of=controller] (output) {};
    \node [block, below of=controller] (measurements) {$H=1$};

    % Once the nodes are placed, connecting them is easy. 
    \draw [draw,->] (input) -- node[pos=0.99] {$+$} node {$V_{s}$} (sum);
    \draw [->] (sum) -- node {$V_{i}$} (controller);
    \draw [->] (controller) -- node [name=y] {$V_{o}$}(output);
    \draw [->] (y) |- (measurements);
    \draw [->] (measurements) -| node[pos=0.99] {$-$} node [near end] {$V_{f}$} (sum);
\end{tikzpicture}
}
	\end{center}
	\caption{}
	\label{fig:ee18btech11012_fig1}
\end{figure}
The transfer function of OPAMP is
\begin{align}
    G\brak{s} = \frac{10^{5}}{(1+\frac{s}{2\pi\times10})}
\end{align}
%
\item For the feedback gain H\\
\solution\\
Feedback gain H can be written as:
\begin{align}
         H = \frac{V_{f}}{V_{o}} = 1
\implies V_{f} = V_{o}
\end{align}
\textbf{Note}:This type of circuit containing Op-amp is called as "Voltage follower" or "Unity buffer".\\
As the non-inverting input of the Op-amp is fed to the output of the system.\\
Which inturn makes the feedback factor(H=1)\\
Here Output ($V_{o}$) follows the input($V_{f}$) as shown in fig.
\item The closed loop transfer function of this system is
\begin{align}
    T = \frac{G(s)}{1+G(s)} = \frac{10^5}{((10^5+1)+\frac{s}{2\pi\times10})}
\end{align}
\item Feedback Circuit for this unity buffer system is
\\
\solution
\begin{figure}[ht!]
	\begin{center}
		\resizebox{\columnwidth}{!}{\begin{circuitikz}[american]

\draw (2,2)  node[op amp] (OA) {};
\draw (OA.+) -- (0,1.5) to[vsourcesin, l= $V_{s}$] (0,0) node[ground](GND){};
\draw (OA.-) -- (0,2.5) node[label={below:$V_{f}$}]{}; 
\draw (OA.out) -- (3,2) node[label={}]{};
\draw (3,2) -- (4.5,2) node[label={above:$V_{o}$}]{};
\draw (3,2) -- (3,4.5) to (0,4.5) -- (0,2.5);
\end{circuitikz}
}
	\end{center}
	\caption{}
	\label{fig:ee18btech11012_fig2}
\end{figure}
\item Verification through Spice circuit
\\
\solution For H=1 the closed loop gain is
\begin{align}
    \abs T \approx \frac{1}{H} = 1
\end{align}
The following is the netlist file for spice simulation
\begin{lstlisting}
spice/ee18btech11012/ee18btech11012.net
\end{lstlisting}
\begin{figure}[!h]
\centering
\includegraphics[width=\columnwidth]{./figs/ee18btech11012_1/ee18btech11012_spiceresult.eps}
\caption{}
\label{fig:ee18btech11012_spiceresult}
\end{figure}
\item The following python code plots the closed loop response verses time and the python plot is also shown below.
\begin{lstlisting}
spice/ee18btech11012/ee18btech11012_spiceresult1.py
\end{lstlisting}
\begin{figure}[!h]
\centering
\includegraphics[width=\columnwidth]{./figs/ee18btech11012_1/ee18btech11012_spiceresult1.eps}
\caption{}
\label{fig:ee18btech11012_spiceresult1}
\end{figure}
\item Checking Unstability in the context of PM \\
\solution A closed loop system is said to be unstable,if the phase margin(PM) of GH is negative\\
\begin{align}
PM < 0\degree\\
\implies \phase{G(f)H(f)} &< -180\degree
\end{align}
For the given GH where H=1
\begin{align}
\phase{G(f)H(f)} = \phase{G(f)} \implies -\tan^{-1}\brak{\frac{f}{10}} 
\end{align}
At $f = \infty$
\begin{align}
\phase{G(f)} =-\tan^{\infty}= -90\degree
\end{align}
So there won't exist any positive f where $\phase{G(f)} < -180\degree$\\
Hence,this system is always stable.

\end{enumerate}
